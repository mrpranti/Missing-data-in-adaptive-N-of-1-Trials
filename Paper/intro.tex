\chapter{Introduction}

The history of N-of 1 trials....\cite{Senarathne2020}
what is n-of-1 trials?

In the world of clinical studies, the interest in personalised medicine and patient-centred healthcare has been increasing. 
 
History of how n-of-1 trials have been developed. 
When to use, and what is the advantage over traditional RCT trials?
Develop towards Bayesian adaptive design.
Its advantages over n-of-1 trials and when to use.

In real-life applications, we always have to consider the missing data issue. In particular, the adaptive N-of-1 trial design depends on the observed values to update the posterior distribution. 
What approaches have been used to tackle N-of-1 trials or SCD?

Are there any approaches or not for adaptive SCD?
If yes, then what approaches have been used for adaptive SCD?
if not, what approaches will be suitable?


Our goal is to find a reliable approach to deal with missing data in Bayesian adaptive N-of-1 trials.

\section{Related Work}
 
we look for related literature where they discussed the isuue of missing data in Thompson sampling or bandit 
problem as the adaptive N-of-1 trials design considered here is based on them. The solution for missing data 
here range from just a simple mean imputation approach to    

\begin{itemize}
    \item Random marker
    \item Mean Imputation \cite{Chen2022}
    \item Regression
    \item Doubly robust estimator ridge regression estimator with pseudo-rewards in\cite{Kim}
    \item unsupervised learning mechanism such as clustering\cite{Bouneffouf2020}
    The authors have considered a scenario where the rewards can be missing in a contextual 
    \item MM-PGPE(GP) is an approach based on the Monte Carlo implementation of the E sep of the EM algorithm.\cite{Wei1990}\cite{Yamaguchi2020}
\end{itemize}

